\chapter*{Prefacio}
Este trabajo tiene por objetivo relacionar la teor'ia de la forma y la teor'ia de espacios finitos. La primera adapta y mejora la teor'ia de homotop'ia en espacios con mal comportamiento local, pasando de considerar las aplicaciones entre espacios a estudiar sucesiones de aplicaciones entre <<aproximaciones regulares>> de dichos espacios, mientras que la segunda relaciona los espacios topol'ogicos finitos con los complejos simpliciales finitos y, en particular, demuestra que sus tipos de homotop'ia d'ebil son iguales. 

En el Cap'itulo \ref{anrs} introducimos los conceptos b'asicos sobre los espacios localmente <<regulares>>, quienes constituir'an los cimientos sobre los que estatuir la teor'ia de la forma. En el Cap'itulo \ref{formahistorica}, haciendo uso de la teor'ia desarrollada previamente, introduciremos la teor'ia de la forma desde  una perspectiva hist'orica, pasando por distintas descripciones de la categor'ia forma hasta llegar a la m'as general en el Cap'itulo \ref{formacategorica}. En 'el, definimos la forma para espacios topol'ogicos cualesquiera y damos algunos ejemplos de aplicaci'on de la teor'ia, yendo desde los invariantes algebraicos hasta las versiones \textit{shape} de algunos teoremas cl'asicos. 

El cap'itulo \ref{espaciosfinitos} revisa los conceptos fundamentales de la teor'ia de espacios finitos, definiendo los funtores <<espacio de caras>> y <<complejo del orden>> que nos llevar'an a establecer la relaci'on entre dicha teor'ia y la teor'ia de la forma. La conexi'on, que se explicita en el Cap'itulo \ref{sucesionesaprox}, pasa por usar estos funtores para construir, a partir de aproximaciones finitas de un espacio compacto m'etrico, sucesiones inversas de poliedros que reflejen la forma del espacio, o sucesiones de espacios finitos cuyo l'imite inverso recupere la topolog'ia del espacio original.


Para concluir el trabajo, incluimos un cap'itulo que implementa un c'odigo que permite calcular, dadas unas ciertas aproximaciones finitas de un compacto m'etrico, los t'erminos de la sucesi'on inversa que lo aproxima. Adem'as, representa dichos t'erminos (que son espacios finitos) en su diagrama de Hasse minimal. 