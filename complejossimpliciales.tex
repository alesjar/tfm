\chapter{Complejos simpliciales}\label{apendice}
En este apéndice vamos a definir brevemente los conceptos relacionados con los complejos simpliciales que aparecen a lo largo del trabajo. 
\begin{definition}
  Un complejo simplicial abstracto $ K $  es una dupla $ (V_K,S_K) $ formada por un conjunto de puntos $ V_K $ a los que llamaremos \textbf{vértices} y un subconjunto $ S_K\subc \pcal(V) $ tal que si $ \sigma\in S_K $, todo $ \tau\subc \sigma $, que llamaremos \textbf{cara} de $ \sigma $, está a su vez en $ S_K $, y además $ \{v\}\in S_K $ para todo $ v\in V_K $. A los elementos de $ S_K $ los llamaremos \textbf{símplices}. Normalmente identificaremos $ K $ con $ S_K $.
\end{definition}
Visto esto, tenemos que estudiar las aplicaciones entre complejos:
\begin{definition}
  Dados dos complejos $ K $ y $ L $, y una aplicación $ \vf:V_K\lra V_L $, decimos que es una \textbf{aplicación simplicial} si para todo $ \sigma \in K $, $ \vf(\sigma)\in L $.
\end{definition}
 
Ahora veamos cómo representarlos en el espacio:
\begin{definition}
  La realización geométrica de un símplice $ \sigma= \{v_0,...,v_n\} $ de un complejo abstracto es el conjunto de combinaciones lineales convexas formales
  \begin{gather*}
    \ol{\sigma } = \left\{\sum_{i=0}^n t_iv_i\mid \sum_{i}t_i = 1, t_i\in [0,1]\right\}.
  \end{gather*}
 Cada s'implice tiene una topolog'ia dada por la m'etrica 
 \begin{gather*}
     \dd(\sum_i t_iv_i ,\sum_i s_iv_i) = \sqrt{\sum_i (t_i-s_i)\sq }
 \end{gather*}
 El \textbf{baricentro} $ b(\sigma) $ de un símplice es el punto resultado de fijar todas las coordenadas $ t_i=\frac{1}{\#\sigma} $.
\end{definition}
\begin{definition}
  La realización geométrica $ \abs{K} $ de un complejo simplicial es la unión de las realizaciones de todos sus símplices con la topología cuyos cerrados son los $ C\subc \abs{K} $ tales que $ C\cap\ol{\sigma } $ es cerrado en $ \ol{\sigma} $ para todo $ \sigma\in K $ (la topolog'ia final con respecto a los $\ol{\sigma}$).
\end{definition}
Si $ K $ es finito, esta coincide con la topología  es la misma que induce la métrica 
\begin{gather*}
  \text{d}(\sum_{v\in K}\alf_v v,\sum_{v\in K}\beta_v v ) = \sqrt[]{\sum_{v\in K}(\alf_v-\beta_v)\sq }.
\end{gather*}

Al espacio m'etrico dado por $(\abs{K},\dd)$ lo denotaremos por $\abs{K}_m$.

\begin{definition}
  Dado un $ x\in \abs{K} $, que podemos representar como $ x = \sum_{v\in K}\alf_v v $ en general, llamamos \textbf{soporte} de $ x $ al símplice $ \{v\mid \alf_v>0\} $ y \textbf{estrella} de un v'ertice $v$ al conjunto $\St(v,\abs{K}) = \{x\in \abs{K}\mid v\in \sop(x)\}$.
\end{definition}

\begin{observation}
  Una aplicación simplicial $ \vf:K\lra L $ induce otra aplicación continua entre las realizaciones geométricas dada por 
  \begin{gather*}
    \begin{matrix}
    \abs{\vf}: \ &\abs{K} &\longrightarrow &\abs{L} \\
    &\sum_{v\in K}\alf_v v &\mapsto &\sum_{v\in K}\alf_v \vf(v).
    \end{matrix}
  \end{gather*}
  Está bien definida porque $\sop(v) $ siempre es un símplice de $ K $, y por tanto $ \vf(\sop(v)) $ lo es de $ L $.
\end{observation}



\begin{definition}\label{subdiv}
  La \textbf{subdivisión baricéntrica} $ K' $ de un complejo $ K $ es el complejo cuyos vértices son los símplices de $ K $ y cuyos símplices son las cadenas de símplices de $ K $. La aplicación lineal dada por
  \begin{gather*}
    \begin{matrix}
    s_K: \ &\abs{K'} &\longrightarrow &\abs{K} \\
    &\sigma  &\mapsto &s_K(\sigma)=b(\sigma) = \sum_{v\in \sigma }\frac{v}{\#\sigma}
    \end{matrix}
  \end{gather*}
  es un homeomorfismo. De hecho, ambas realizaciones se pueden construir de tal forma que sean iguales y $ \abs{s_K }$ sea la identidad.
\end{definition}
\begin{proposition}\label{subdivbar}
  Si escogemos como vértices de la realización de $ K'  $ los baricentros de los símplices de $ K  $, entonces $ \abs{K' } = \abs{K } $ como espacios topológicos y $s_K $ es la identidad.
\end{proposition}
\begin{proof}
  Si realizamos $ K  $ con un conjunto de vértices $ \{v_i\}_{i\in I}  $, cada símplice de $ K'  $ vendrá dado por unos  vértices $ \{b_{\sigma_0 },...,b_{\sigma_m }\} $ donde $ \sigma_0\subset \sigma_1\subset...\subset \sigma_m  $ es una cadena de símplices de $ K  $. Desde luego, todos los vértices de posibles símplices de $ K'  $ son puntos en la realización de un mismo símplice de $ K  $, y por ende toda realización de un símplice de $ K'  $ está contenida en una de un símplice  de $K  $. 
  
  Recíprocamente, queremos ver todo punto $x$ en la realización de un símplice $ \sigma = \{v_0,...,v_m \}\in K  $ está también en la realización de uno de $K'$. Si $ x\in \ol{\sigma } $, pongamos $ x = \sum_{i=0}^m t_iv_i $ y reordenemos y eliminemos vértices de tal forma que  $ t_0\geq t_1\geq...\geq t_m >0 $.  Queremos reescribir $ x  $ como combinación convexa de vértices de $K'$ que formen un símplice, es decir, de baricentros de una cadena de símplices de $K$. Observamos, según la intuición geométrica, que los baricentros que intervengan en la expresión de nuestro punto tendrán que ser los asociados a las caras más próximas al punto, y estas distancias vienen representadas por las coordenadas respecto a los vértices. De esta forma, si denotamos $ \tau_0 = \{v_0\},...,\tau_{m-1} = \{v_0,...,v_{m-1}\},\tau_m = \sigma  $, podemos plantear la ecuación 
  \begin{gather*}
    x = \sum_{i=0}^m t_i v_i = \alf_0 b_{\tau_0}+...+\alf_m b_{\tau_m}.
  \end{gather*}
  Sustituyendo cada baricentro por su valor, llegamos a las ecuaciones 
  \begin{gather*}
    \sum_{i=0}^m t_iv_{i} = \sum_{i=0}^m \left(\frac{\alf_i }{i+1}+...+\frac{\alf_m}{m+1}\right)v_i,
  \end{gather*}
  y reescribiéndolo como sistema, 
  \begin{gather*}
    \left\{ \begin{matrix}
        \alf_0 + & \frac{\alf_1 }{2} +...+&\frac{\alf_m }{m+1} = t_0\\ 
      & \frac{\alf_1 }{2} +...+&\frac{\alf_m }{m+1} = t_1 \\ 
      &&\vdots \\ 
      & & \frac{\alf_m }{m+1} = t_m
    \end{matrix}\right.
  \end{gather*}
  Este sistema tiene solución única y esta es $\alf_i = (t_{i}-t_{i+1})(i+1) \geq 0$. Sumando las ecuaciones del sistema se desprende también que $\sum_i \alf_i = \sum_i t_i =1 $ y podemos escribir $ x  $ como combinación de vértices de $ K'  $, de donde $ \abs{K' } = \abs{K } $ como conjuntos. En cuanto a la topología, si $ C$ es cerrado en $ \ol{\sigma } $, toda $ C\cap \ol{\tau } = C\cap \ol{\sigma }\cap \ol{\tau }$ es cerrada por ser intersección de cerrados, así que todo cerrado de $ \abs{K}  $ lo es de $ \abs{K' } $. Recíprocamente, todo $ \ol{\sigma } \subc \abs{K} $ es unión de una cantidad finita de $ \ol{\tau }\subc \abs{K' } $, y por tanto $ C\cap \ol{\sigma} = \bigcup_\tau C\cap\ol{\tau } $ es cerrada.
\end{proof}
Enunciamos ahora un lema que nos servir'a a para probar la siguiente proposici'on:
\begin{lemma}\label{idhomotopia}
    La identidad $\id : \abs{K}\lra \abs{K}_m$ es una equivalencia homot'opica.
\end{lemma}
\begin{proof}
    La prueba se puede encontrar en \cite[p. 302]{mardešić1982shape}.
\end{proof}

A lo largo de todo el trabajo haremos uso del siguiente criterio para probar que las realizaciones de dos aplicaciones simpliciales son hom'otopas:
\begin{definition}
  Diremos que dos aplicaciones continuas $ g,f: X \lra \abs{K } \ (\abs{K}_m) $ de un espacio topológico en la realización de un complejo simplicial son \textbf{simplicialmente cercanas} si, para todo $ x\in X  $, las imágenes $ f\px $ y $ g\px  $ pertenecen a la realización de un $ \sigma \in K  $.
\end{definition}
\begin{proposition}\label{simplicialmentecercanas}
  Dos aplicaciones simplicialmente cercanas son siempre homótopas.
\end{proposition}
\begin{proof}
  Hagamos primero el caso m'etrico. Por ser simplicialmente cercanas, la aplicaci'on
  \begin{gather*}
    \begin{matrix}
    H: \ &X \times I &\longrightarrow &\abs{K}_m \\
    &(x,t) &\mapsto &tf\px +(1-t)g\px
    \end{matrix}
  \end{gather*}
  está bien definida. La continuidad es sencilla gracias a la m'etrica:
  \begin{gather*}
      \dd(H(x,t),H(x_0,t_0)) = \dd(tf(x)+(1-t)g\px, t_0f(x_0)+(1-t_0)g(x_0)) = \\ 
      \dd(tf(x)+(1-t)g\px,tf(x_0)+(1-t)g(x_0)) +\\ 
      \dd(tf(x_0)+(1-t)g(x_0),t_0f(x_0)+(1-t_0)g(x_0)), 
  \end{gather*}
   donde las distancias anteriores se pueden reducir arbitrariamente por la continuidad de $f$ y $g$. En el caso de estar en la topolog'ia final, el Lema \ref{idhomotopia} y el caso m'etrico nos aseguran la homotop'ia $\id \circ f\simeq \id \circ g$. Si llamamos $j$ a la inversa homot'opica de la identidad, 
   \begin{gather*}
       f\simeq j\circ \id\circ  f\simeq j \circ \id \circ g \simeq g,
   \end{gather*}
   concluyendo la demostraci'on.
\end{proof}


\begin{definition}
  Dadas aplicaciones simpliciales entre complejos  $\vf,\psi :K\lra L$, diremos que son \textbf{contiguas} si para cada $ \sigma\in K $ se tiene $ \vf(\sigma)\cup \psi(\sigma)\in L $.
\end{definition}

\begin{proposition}\label{contiguashomotopas}
  Si dos aplicaciones simpliciales $ \vf  $ y $ \psi  $ son contiguas, sus aplicaciones asociadas entre las realizaciones son homótopas. 
\end{proposition}
\begin{proof}
  Tenemos dos aplicaciones $ \abs{\vf},\abs{\psi}:\abs{K}\lra\abs{L} $ continuas tales que dado $ x\in \abs{K} $, será $ x\in \ol{\sigma } $ para algún $ \sigma\in K $. Entonces $ \abs{\vf(x)}$ y $ \abs{\psi(x)}$ pertenecen a $ \ol{\vf(\sigma)\cup\psi(\sigma ) }$ para todo $ x\in \abs{K}$, es decir, son simplicialmente cercanas y por tanto homótopas.
\end{proof}
\iffalse
\begin{definition}
  Un \textbf{cono simplicial} es un complejo $ K $ que tiene un vértice $ a $ tal que para todo $ \sigma\in K $, $ \sigma\cup \{a\} $ también es símplice de $ K $.
\end{definition}
\begin{proposition}\label{conosimp}
  Si $ K $ es un cono simplicial, $ \abs{K} $ es contráctil.
\end{proposition}
\begin{proof}
  Basta usar la proposición anterior: la identidad es contigua a la aplicación que envía cada vértice a $ a $.
\end{proof}
\fi